%!TEX TS-program = xelatex 
%!TEX TS-options = -synctex=1 -output-driver="xdvipdfmx -q -E"
%!TEX encoding = UTF-8 Unicode
%
%  rarefaction
%
%  Created by Mark Eli Kalderon on 2018-03-29.
%  Copyright (c) 2018. All rights reserved.
%

\documentclass[12pt]{article} 

% Definitions
\newcommand\mykeywords{audition, sound, source, wave theory}
\newcommand\myauthor{Mark Eli Kalderon}
\newcommand\mytitle{The Event of Rarefaction}

% Packages
\usepackage{geometry} \geometry{a4paper} 
% \usepackage{txfonts}
\usepackage{enumerate}
\usepackage{setspace}
\doublespace % Uncomment for doublespacing if necessary
% \usepackage{epigraph} % optional

% XeTeX
\usepackage[cm-default]{fontspec}
\usepackage{xltxtra,xunicode}
\defaultfontfeatures{Scale=MatchLowercase,Mapping=tex-text}
\setmainfont{Hoefler Text}

% Bibliography
\usepackage[round]{natbib}

% Title Information
\title{\mytitle}
\author{\myauthor}
\date{} % Leave blank for no date, comment out for most recent date

% PDF Stuff
\usepackage[plainpages=false, pdfpagelabels, bookmarksnumbered, backref, pdftitle={\mytitle}, pdfauthor={\myauthor}, pdfkeywords={\mykeywords}, xetex, unicode=true]{hyperref} 

%%% BEGIN DOCUMENT
\begin{document}

% Title Page
\maketitle
% \vskip 2em \hrule height 0.4pt \vskip 2em
% \epigraph{} % optional; make sure to uncomment \usepackage{epigraph}

% Layout Settings
\setlength{\parindent}{1em}

% Main Content

\section{Motion in a Medium} % (fold)
\label{sec:motion_in_a_medium}

In this paper I shall defend and develop The Wave Theory of Sound. The Wave Theory has come on hard times of late---unfairly, to my mind. Many are of the opinion that it is no longer a viable alternative in the metaphysics of sound, and this despite a long and venerable history. As we shall see, many apparent difficulties facing The Wave Theory are due to an exclusive focus on hearing sounds. An exclusive focus on hearing sounds can lead to a distorted conception not only of audition but of the sounds themselves. With respect to audition it can encourage the thought that the function of audition is to afford the perceiver with auditory awareness of sounds. But if we can hear not only sounds, but their sources, understood as sound-generating events, it is more likely that the function of audition is to afford the perceiver awareness of the distal sources of sound. With respect to sound, an exclusive focus on hearing sounds can lead the conflation of features of sources of sound with features of the sounds themselves. Many of the recent criticisms of The Wave Theory turn on just such a conflation.

% section motion_in_a_medium (end)

\section{Assumptions} % (fold)
\label{sec:section_name}



% section assumptions (end)

\section{Sonic Events} % (fold)
\label{sec:sonic_events}



% section sonic_events (end)

\section{Motive and Objections} % (fold)
\label{sec:motive_and_objections}



% section motive_and_objections (end)

\section{The Event of Rarefaction} % (fold)
\label{sec:the_event_of_rarefaction}



% section the_event_of_rarefaction (end)

% Bibligography
\bibliographystyle{plainnat} 
\bibliography{mybib} 

\end{document}