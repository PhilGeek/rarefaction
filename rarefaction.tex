%!TEX TS-program = xelatex 
%!TEX TS-options = -synctex=1 -output-driver="xdvipdfmx -q -E"
%!TEX encoding = UTF-8 Unicode
%
%  rarefaction
%
%  Created by Mark Eli Kalderon on 2018-03-29.
%  Copyright (c) 2018. All rights reserved.
%

\documentclass[12pt]{article} 

% Definitions
\newcommand\mykeywords{audition, sound, source, wave theory}
\newcommand\myauthor{Mark Eli Kalderon}
\newcommand\mytitle{The Event of Rarefaction}
\newcommand\mysubtitle{A Defence and Development of The Wave Theory of Sound}

% Packages
\usepackage{geometry} \geometry{a4paper} 
% \usepackage{txfonts}
\usepackage{enumerate}
% \usepackage{setspace}
% \doublespace % Uncomment for doublespacing if necessary
% \usepackage{epigraph} % optional

% XeTeX
\usepackage[cm-default]{fontspec}
\usepackage{xltxtra,xunicode}
\defaultfontfeatures{Scale=MatchLowercase,Mapping=tex-text}
\setmainfont{Hoefler Text}

% Bibliography
\usepackage[round]{natbib}

% Title Information
\title{The Event of Rarefaction:\\A Defence and Development of The Wave Theory of Sound}
\author{\myauthor}
\date{} % Leave blank for no date, comment out for most recent date

% PDF Stuff
\usepackage[plainpages=false, pdfpagelabels, bookmarksnumbered, backref, pdftitle={\mytitle}, pdfauthor={\myauthor}, pdfkeywords={\mykeywords}, xetex, unicode=true]{hyperref} 

%%% BEGIN DOCUMENT
\begin{document}

% Title Page
\maketitle
\begin{abstract} % optional
\noindent I defend and develop a traditional view in the metaphysics of sound, The Wave Theory of Sound. According The Wave Theory, as developed herein, sounds are not patterned disturbances so much as their propagation. And the propagation of a patterned disturbance is not a form of travel, but a dynamic in-formation, the wave-form successively inhering in differently located parts of the dense and elastic medium. This conception, along with the assumption that we hear not only sounds but their sources, has the resources to address many of the most recent criticisms of this traditional view.
\end{abstract}
% \vskip 2em \hrule height 0.4pt \vskip 2em
% \epigraph{} % optional; make sure to uncomment \usepackage{epigraph}

% Layout Settings
\setlength{\parindent}{1em}

% Main Content

\section{Motion in a Medium} % (fold)
\label{sec:motion_in_a_medium}

In this essay I shall defend and develop The Wave Theory of Sound. According to The Wave Theory, sound is a kind of motion in a medium, specifically, an event of rarefaction. What this means exactly requires some unpacking. Providing a credible and coherent interpretation of The Wave Theory is the central aim of the present essay. 

The Wave Theory has come on hard times of late---unfairly, to my mind. Many are of the opinion that it is no longer a viable alternative in the metaphysics of sound, and this despite a long and venerable history (for some of that history see \citealt{Pasnau:2000aa}). As we shall see, many apparent difficulties facing The Wave Theory are due to an exclusive focus on hearing sounds. An exclusive focus on hearing sounds can lead to a distorted conception not only of audition but of the sounds themselves. With respect to audition it can encourage, for example, the thought that the function of audition is to afford the perceiver with auditory awareness of sounds. But if we can hear not only sounds, but their sources as well, understood as sound-generating events, it is more likely that the function of audition is to afford the perceiver awareness of the distal sources of sound (\citealt{Nudds:2009sf}, \citealt{Leddington:2014aa}, \citealt[chapter 4]{Kalderon:2018oe}). With respect to sound, an exclusive focus on hearing sounds can lead to conflating features of sources with features of the sounds that they generate. Many of the recent criticisms of The Wave Theory turn on just such a conflation, as do many of the arguments in favor of so-called distal conceptions of sound.

The aim of this essay is not to establish The Wave Theory of Sound from first principles. Nor even that it is better, on balance, than its alternatives. Rather, in defending and developing The Wave Theory of Sound, I aim only to establish that The Wave Theory is a coherent, credible alternative in the metaphysics of sound. In aiming only to establish the intelligibility of The Wave Theory of Sound, many of the critical remarks about its alternatives are not meant to persuade their adherents so much as to show how these alternatives might intelligibly be resisted within a given framework.

% section motion_in_a_medium (end)

\section{Assumptions} % (fold)
\label{sec:section_name}

In defending and developing The Wave Theory of Sound, I shall make a number of assumptions. The conclusions of the present essay should be understood to be conditional upon them:
\begin{enumerate}[(1)]
	\item Sounds are particulars.
	\item Sounds are the bearers or \emph{substrata} of audible qualities.
	\item (With Berkeley) if sources are audible, then sources have audible qualities.
	\item Sounds are not the sole bearers or \emph{substrata} of audible qualities; in propitious circumstances, we can hear, as well, their sources (understood as sound-generating events).
	\item The bearers of audible qualities are events, and hence audible qualities are essentially sustained by activity.
\end{enumerate}
Though I cannot here make a full dress defence, allow me to both explain and motivate these insofar as I can.

(1) \emph{Sounds are particulars.} Though I do not assume it here, I am tempted by the more general claim that all sensibles are particulars. (Very roughly, if perception is best conceived as an encounter with the sensible, and one can only encounter particulars, then the sensibles that we encounter in perceptual experience must themselves be particulars.) That sounds are particulars is a substantive claim, though in some ways a liberal one. Particulars, after all, belong to a range of ontological categories. Thus bodies, events, processes, and quality instances are all particulars, but they have different modes of being and thus belong to different ontological categories. But if particulars belong to a range of ontological categories then the claim that sounds are particulars does not yet determine which ontological category sounds belong to. Thus while a substantive claim, it is liberal in the sense of being consistent with substantively different metaphysics of sounds. 

For example, that at least heard sounds are particulars is consistent even with a broadly Lockean metaphysics of sound where sounds are qualities, if not necessarily secondary qualities (besides Locke's \emph{An Essay Concerning Human Understanding} 2 8, see \citealt{Pasnau:1999ss}, \citealt{Kulvicki:2008aa}, \citealt{Cohen:2010ax}, and \citealt{Roberts:2017as}). Hearing is objective. If there is nothing actual heard, then nothing is actually heard. So if sounds are qualities, then at least in the case of heard sound, the sound that you hear must be a quality instance and not an uninstantiated universal. Only instances of universals and not uninstantiated universals are actual in the relevant sense. And if the quality is conceived as a power or disposition, then it must be their manifestation. Only manifestations of powers or dispositions are actual in the relevant sense (in Peripatetic vocabulary, a second actuality, \emph{De Anima} 2 5). So at least with respect to heard sound, our first assumption is consistent even with a broadly Lockean metaphysics of sound. 

(2) \emph{Sounds are the bearers or \emph{substrata} of audible qualities.} Our first assumption may be consistent with a broadly Lockean metaphysics of sound, at least in the case of heard sound, but not so our second assumption, for according to it, when combined with the first, sounds are not qualities, but the particulars that are the bearers or \emph{substrata} of audible qualities. Aristotle, in \emph{De Anima} 2 11 422 b 31--2, endorses this view on phenomenological grounds. It is at least the case that we are inclined to describe our experience of sounds in this way. Thus we say that we can hear a sound getting louder, or its pitch varying over time. Auditory experience seems to present sounds with a dynamic profile of auditory qualities. The audible qualities of a sound thus vary over time, and the sound persists through this variation. Indeed, the kind of sound that it is depends upon the pattern of variation in audible qualities that it displays. (For discussion of these points, see \citealt{OCallaghan:2010aa}.) Given our assumptions, sounds may be particulars that are the bearers or \emph{substrata} of audible qualities, and so not audible qualities themselves, but so far at least it remains indeterminate which ontological category these particulars belong to. The ontological category of these particulars will be addressed by our fifth assumption.

(3) \emph{(With Berkeley) if sources are audible, then sources have audible qualities.} According to Berkeley, we hear only sounds. The sources of sounds are, strictly speaking, inaudible. In \emph{Three Dialogues between Hylas and Philonous}, Berkeley distinguishes sounds from their sources by an application of Leibniz's Law. Sounds have audible qualities that sources lack. Sources, lacking audible qualities, are thereby inaudible. So Berkeley is assuming that if sources are audible, then they have audible qualities. He denies that they have audible qualities and concludes, by \emph{modus tollens}, that sources are inaudible. Equally, however, on may grant Berkeley the conditional and reason instead my \emph{modus ponens}. Doing so would be to assume, as our next assumption does, that sources are audible and conclude that they have audible qualities. What audible qualities could they have? Insofar as it is associated with the material structure of bodies  relevant to their participating in the sound-generating event, timbre is a good candidate. Or if timbre is not a well defined category, if it is understood, in a deflationary fashion, as whatever varies as pitch and volume are held constant, then some relatively natural subcomponent of timbre is a good candidate for being an audible quality of the source understood as a sound-generating event (though see \citealt{Kulvicki:2008aa} for an argument that timbre is a quality of bodies and not their activity).

Importantly, the present assumption does not presuppose that only audible qualities are heard. That would be tantamount to a broadly Lockean metaphysics of sound coupled with the Berkelean claim that only sounds are heard. Sounds are heard. And the identity of a heard sound depends upon is duration and the variation in audible qualities that it manifests. Moreover, the duration and variation in audible qualities of a heard sound can themselves be heard. But duration is not an audible quality like pitch or timbre, it is a quantity. And the variation of audible qualities is not itself an audible quality but their variation over time. So it is not the case that only audible qualities are heard. Nevertheless, the variation of a sound's audible qualities could only be heard if the varying audible qualities were themselves heard. And the duration of a sound could only be heard, if the audible qualities that it manifests over its temporal interval are heard. And so similarly, sources are only heard in hearing their audible qualities.

(4) \emph{Sounds are not the sole bearers or \emph{substrata} of audible qualities; in propitious circumstances, we can hear, as well, their sources (understood as sound-generating events).} I shall assume that we not only hear sounds but, at least in propitious circumstances, their sources as well. We ordinarily speak of bodies and their sound-generating activities indifferently as sources. Bodies are said to be sources insofar as they have the power to sound, to engage in sound-generating activity. However, by source, I shall mean, not a body possessing the power to sound, so much as the sound-generating activity that it gives rise to. For present purposes I restrict talk of sources to sound-generating events since plausibly only these, and not bodies, figure in auditory experience (for a defence of this and a partial accommodation of the opposing view see \citealt[chapter 3.4]{Kalderon:2018oe}). Sources, so understood, are sound-generating events. If we hear, not only sounds, but sometimes their sources as well, then sounds are not alone in being the bearers or \emph{substrata} of audible qualities. If we sometimes hear the sources of sound---not just the sound of the door slamming but the door slamming---then the sources of sound are audible. And if they are audible, then they must possess audible qualities. Again, a natural subcomponent of timbre, since it carries information about the material structure of bodies relevant to their participation in the sound-generating event, is a good candidate for an audible quality of a sound source.

That we can hear not only sounds but their sources as well is an important thesis in the philosophy of audition since it bears not only on the objects of auditory experience but on its nature as well. It bears on the nature of auditory experience since, as we have observed, it raises a question about its function: Is the function of auditory experience to afford awareness of sounds or their sources? Not only does it bear on the function of auditory experience but potentially on its structure as well. Someone who thinks that we hear sources by hearing the sounds that they generate will conceive of the structure of auditory experience differently from someone who denies this (a topic explored in section~\ref{sec:hearing_sources_in_the_sounds_they_generate}). And as it bears on the objects of auditory experience, it is also an important thesis in the metaphysics of sound. If we can hear not only sounds but their sources, then having identified that an object of auditory experience has some feature, one may not automatically conclude that it is a feature of a sound---after all, it may be a feature of its heard source. So an exclusive focus on hearing sound, can lead to a distorted conception, not only of audition, but of the sounds themselves.

(5) \emph{The bearers of audible qualities are events, and hence audible qualities are essentially sustained by activity.} Sources, in our restricted sense, are events. They are, after all, sound-generating events. But what ontological category do sounds belong to? Are they too events? Reflection upon the conditions under which audible qualities are instantiated suggests that they are. Insofar as sounds and their sources are the bearers or \emph{substrata} of audible qualities, sounds, like their sources, must be essentially dynamic entities like events. Audible qualities are only ever instantiated by things with duration. We can imagine hearing briefer and briefer pitches but we cannot conceive of instantaneous pitch, pitch without duration. Sounds, insofar as they are audible, essentially have duration then. (Prichard's \citeyear{Prichard:1950ly} inaugural \emph{aporia} concerning the inaudibility of sound turns on this point.) Not only do they have a beginning and end, like mortal animals and other bodies, but they have a distinctive way of being in time. A sound is not wholly present at each moment of its sounding. Rather, a sound unfolds through the temporal interval of its sounding. The sound's audible qualities vary and extinguish as its activity varies and extinguishes, and what makes it the sound that it is is just this duration and pattern of dynamic variation. Sounds thus have the same temporal mode of being as events (at least as a three-dimensionalist conceives of them, see \citealt{Fine:2006fk}, for criticism see \citealt{Sider:1997fk} and \citealt{Hawthorne:2008uq}). 

If both sounds and their sources are events, and these are the only bearers or \emph{substrata} of audible qualities, then the bearers or \emph{substrata} of audible qualities belong to a uniform ontological category, they are essentially dynamic entities. Conversely, this reveals something about the nature of audible qualities. Audible qualities are qualities essentially sustained by activity. The audible qualities of a sound will vary and extinguish as the sound’s activity varies and extinguishes. Some states and qualities are essentially sustained by activity. Thus \citet{Ryle:1949qr} gives the example of keeping the enemy at bay. The enemy being kept at bay is a state of the larger battle, a stable phase of it, but one that requires military activity to sustain. And \citet{kripke72} gives the example of heat and its connection with molecular motion. Heat is a thermal quality, and a sensible one, but it is a sensible quality that requires molecular activity to sustain. The only bearers of audible qualities present in auditory experience are essentially dynamic entities. While a thesis about the kind of particulars that can be present in auditory experience, when coupled with our assumption that these particulars are the bearers or \emph{substrata} of audible qualities, it has an important consequence for the nature of audible qualities. If audible qualities only ever inhere in essentially dynamic entities, then they are qualities essentially sustained by activity. 

% section assumptions (end)

\section{Sonic Events} % (fold)
\label{sec:sonic_events}

If sounds are events, then what events are they? Consider the following three candidates:

\begin{enumerate}[(1)]
	\item Sounds are events that would cause a patterned disturbance to propagate, in every direction, along rectilinear paths, through a dense and elastic medium, should there be one. \citep{Casati:1994aa}
	\item Sounds are the causing of a patterned disturbance to propagate, in every direction, along rectilinear paths, through a dense and elastic medium. \citep{OCallaghan:2007xy}
	\item Sounds are events of rarefaction. Specifically, they are the propagation, in every direction, along rectilinear paths, of a patterned disturbance, through a dense and elastic medium. (The Wave Theory, \citealt{OShaughnessy:2009aa}, \citealt{Sorensen:2009aa}, \citealt[chapters 3–4]{Kalderon:2018oe})
\end{enumerate}
Our three candidate events have a number of elements in common. It is unsurprising that they do given that they are related, if distinct, events. With respect to the first two candidate events, the common elements are features of their characteristic effect. With respect to the third candidate, the common elements are, by contrast, features of the event itself. What then are their common elements? The patterned disturbance consists in longitudinal compression waves that vary in amplitude and frequency. These occur in a dense and elastic medium, be it air, or water, or some other suitable material. The patterned disturbance is propagating through the dense and elastic medium. (The nature of the pattern disturbance's propagation---whether it is locomotion or some other species of change---will be addressed in the subsequent section.) Moreover, the propagation of the patterned disturbance has a certain dynamic structure. The patterned disturbance is propagating, in every direction, along rectilinear paths, from its source. The force with which the patterned disturbance propagates diffuses through the dense and elastic medium. 

Roger Bacon observes that this is characteristic of events of rarefaction:
\begin{quote}
	Sound is generated when the parts of an object that had been struck from their natural position, and in this place there comes to be a vibration of parts [of the surrounding air] in all directions, accompanied by a certain rarefaction, since the motion of rarefaction is [along a line] from the centre to the circumference. (Roger Bacon, \emph{Perspectiva} 1 8 2 82-85; \citealt{Lindberg:1996bk})
\end{quote}
Influenced by al-Kindī and Robert Grosseteste, Bacon’s doctrine was modeled on the propagation of light:
\begin{quote}
	For light of its very nature diffuses itself in every direction in such a way that a point of light will produce instantaneously a sphere of light of any size whatsoever, unless some opaque object stands in the way. (Robert Grosseteste, \emph{De Luce}; \citealt[10]{Riedl:1942it})
\end{quote}

The first candidate event, one that would cause a patterned disturbance to propagate, in every direction, along rectilinear paths, through a dense and elastic medium should there be one, is existentially independent of the medium. It is also heterogenous. It is the bowing, grinding, scraping, vibrating, or whatever kind of event that would cause a patterned disturbance to propagate, in every direction, along rectilinear paths, through a dense and elastic medium, should there be one. But these events could occur even in the absence of such a medium.

The second candidate event, the causing of the patterned disturbance to propagate, in every direction, along rectilinear paths, through a dense and elastic medium, is not existentially independent of the medium nor is it heterogenous. The event of causing a patterned disturbance to propagate through a medium could only occur if there were such a medium. The second candidate event existentially depends upon the medium. This difference between the first two candidates emerges in the interpretation they give to the tag line of the 1979 movie \emph{Aliens}: ``In space no one can hear you scream.'' On the first candidate, sounds are existentially independent of the medium, and so the problem is the perceptual inaccessibility of your screams since there is no suitable medium in space to convey them to an audience. The second candidate, by contrast, endorses a conception of sound that is existentially dependent upon the medium, and so the problem is the absence of your screams in space, and this despite your best efforts to produce them. (For discussion see \citealt{Pasnau:1999ss}, \citealt{OCallaghan:2007xy,OCallaghan:2009aa}) The second candidate differs from the first, not only in its existential dependence upon a medium, but also in being non-heterogenous. The causing of a patterned disturbance to propagate through a medium has a more unified structure then the bowing, grinding, scraping, vibrating, and so on (though these would be causings should they occur in such medium).

The third candidate event identifies sounds, not with patterned disturbances, but with their propagation. Why identify sounds with the propagation of patterned disturbance rather than with the patterned disturbance itself? The force with which the patterned disturbance propagates weakens as it diffuses through the dense and imperfectly elastic medium. As a consequence the patterned disturbance deforms as it propagates. Though it changes as it propagates, the patterned disturbance does so in predictable ways. Ordinarily we are happy to say that two perceivers, one near to the source and one far, can both hear the sound that it generates and indeed the same sound even though heard near it sounds louder than when heard far. If sounds were patterned disturbances and not their propagation, we would be forced to conclude that our two perceivers hear different sounds, since the patterned disturbance that was the proximal cause of their hearing differed in each case. For this reason, The Wave Theory is better understood as the thesis that sounds are the propagation of a patterned disturbance rather than the patterned disturbance. Doing so better coheres with our ordinary criteria for individuating sound. There is an important additional reason as well. As we shall see in section~\ref{sec:motive_and_objections}, identifying sounds with the propagation of a patterned disturbance better coheres with emanative phenomenology of auditory experience.

The third candidate event, like the second, existentially depends upon the me\-di\-um. Like the second candidate, in space no one can hear you scream in the sense that no scream is produced despite your best efforts. According to The Wave Theory, at least as developed herein, sounds are the propagation, in every direction, along rectilinear paths, of a patterned disturbance, through a dense and elastic medium. Such events of could only occur if a suitable medium exists. Like the second candidate, the third candidate event existentially depends upon the medium. Also like the second candidate, the third candidate has a unified structure and so is non-heterogenous. The unified structure of the third candidate is less abstract and more dynamic than the unified structure of the second candidate. As Bacon observed, it has the characteristic structure of an event of rarefaction. The force with which the patterned disturbance propagates diffuses through the dense and elastic medium. And diffusion occurs along rectilinear paths from the center to the expanding circumference of a sphere. Assuming the medium is perfectly elastic, and ignoring density and potential obstructors of sound, we can envision the event of rarefaction as an ever expanding sphere, the patterned disturbance occurring at its outer boundary. 

Neither the first nor the second candidate events display the dynamic structure of an event of rarefaction. The first candidate event could occur or not emdedded in a suitable medium. Events of rarefaction, however, do not merely occur in a medium, they are what happens to the medium. Moreover, the dynamic structure of what happens to the medium is not reflected in the bowing, grinding, scraping, vibrating, or whatever kind of event that disturbs the medium. There is no sense in which the grinding, for example, is propagating, in every direction, along rectilinear paths. The second candidate may be more unified than the first in identifying sounds with the causing of patterned disturbances to propagate, in every direction, along rectilinear paths, through a dense and elastic medium, but the dynamic structure is what the causing effects, and is not a feature of the causing itself. The causing does not itself, for example, propagate, in every direction, along rectilinear paths. That is a feature of the effect, not the cause.

From the perspective of The Wave Theory of Sound, the first candidate event is not a sound, but a potential source of sound. Bowing, grinding, scraping, vibrating, and so on, are the kind of events that would generate a sound should they occur in a dense and elastic medium. They are potential sound-generating events where this potential is realized only when they occur in a suitable medium. If The Wave Theory is true, if sounds are events of rarefaction, then identifying sounds with events that would cause a patterned disturbance to propagate, in every direction, along rectilinear paths, through a dense and elastic medium, should there be one simply conflates sounds with their sources. The second candidate event, unlike the first, does not simply misidentify sound with the source that generates it. However, as we shall see, it misattributes features of the source to the sound that it generates. From the perspective of The Wave Theory of Sound, this is a general tendency among distal conceptions of sound, conceptions according to which sounds have determinate locations remote from the perceiver. 

% section sonic_events (end)

\section{Two Models of Sonic Propagation} % (fold)
\label{sec:two_models_of_sonic_propagation}

Plato and Aristotle identify sound with motion in a medium (\emph{Timaeus} 67a–c, 80a–b; \emph{De Anima} 2 8). Though they identify sound with motion in a medium we are still a long way from The Wave Theory as defended herein (see, for example, Cornford's \citeyear[320 n.1]{Cornford:1935fk} criticism of \citealt{Archer-Hind:1888qd}). For one thing, both Plato and Aristotle conceive of the medium being moved as a continuous unity (on this point, in the \emph{Timaeus} account of sound, see \citealt[109]{Beare:1906uq}). Moreover, the view may be differently developed depending upon what, exactly, motion---\emph{kìnêsis}---means. \emph{kìnêsis} is Aristotle's general term for change of any kind, and so not merely locomotion. And though it is not a technical term in Plato's writing the way that it is Aristotle's, a similar usage can be found especially in the \emph{Timaeus}. This is important since \emph{kìnêsis} here could be locomotion, a change of place over time, or it could instead be a species of alteration. These contrasting interpretative hypotheses provide us with two models of sonic propagation.

On the first model, sonic propagation is understood in terms of locomotion. So understood, the patterned disturbance is literally travelling through the dense and elastic medium. Locomotion is a change paradigmatically undergone by bodies. In the case of Plato and Aristotle, the relevant bodies would be continuous bodies of air produced by a sharp percussive blow. However, since we are assuming that sounds are events and not bodies, does that mean that the model is inapplicable from the start? The issue is unobvious since events have locations, though perhaps not in the way that bodies do (see, for example, \citealt{Davidson:1969da}). And, sometimes at least, events change their locations over time. Thus the fight started in the bar and spilled out into the street. Still, if it is specifically the patterned disturbance that is in ``motion'', and not the dense and elastic medium that instantiates the patterned disturbance, then the propagation of the patterned disturbance is better understood as a species of alteration.

For this reason, Prichard denies that waves and sounds, being what they are, are subject to locomotion. Only bodies move, and waves and sounds are not bodies:
\begin{quotation}
	But \dots\ I also made the same remark (viz.\ that only a body could move) to a mathematician here. What was in my mind was that it is mere inaccuracy to say that a wave could move, and that where people talk of a wave as moving, say with the velocity of a foot, or a mile, or 150,000 miles, a second, the real movement consisted of the oscillations of certain particles, each of which took place a little later than a neighbouring oscillation.
	
	He scoffed for quite a different reason. He said that you could illustrate a movement by a noise---that, for example, if an explosion occurred in the middle of Oxford the noise would spread outwards, being heard at different times by people at varying distances from the centre, so that at one moment the noise was at one place and that a little later it was somewhere else, and in the interval it had moved from one place to the other.
	
	Now, of course, it was not in dispute that in the process imagined people in different places each heard a noise at a rather different time. The only question was, `Was the succession of noises a movement?', and I think that on considering the matter you will have to allow that it was not, and that what happened was that he, being certain of the noises, and wanting to limit the term `movement' to something he was certain of, used the term `movement' to designate the succession of noises, implying that this was the real thing of which we were both talking. But if this is what happened, then he was using the term `movement' in a sense of his own, and in saying that in the imagined case he was certain of a movement, he was being certain of something other than the opposite of what I was certain of. \citep[99]{Prichard:1950kx}
\end{quotation}
\citet[430 n. 29, appendix]{Burnyeat:1995fk} lampoons Aristotle for making similar claims by citing Prichard echoing them here. I have argued, that at least in this instance, Burnyeat is hoisted by his own petard, \citep[chapter 3.2]{Kalderon:2015fr}. If Prichard is right, then sonic propagation cannot be understood in terms of locomotion.

On the second model, sonic propagation is understood, not in terms of locomotion, but in terms of alteration. So understood, sonic propagation is a kind of dynamic in-formation. Specifically, the patterned disturbance is successively in-forming differently located parts of the dense and elastic medium along a rectilinear paths in every direction away from the source of the sound. Thus the source's sounding will initially produce a patterned disturbance in the dense and elastic medium adjacent to the source. At a later moment, the patterned disturbance will be instantiated by a different part of the dense and elastic medium, further away from the source along a rectilinear path. On this model, the patterned disturbance is not travelling, at least not literally. It is not the case that the patterned disturbance is now located here and now there, at least not directly (\emph{De Anima} 1 3 406 a 3--8). Rather, differently located parts of the dense and elastic medium come to successively instantiate the patterned disturbance. A change of state and travel are distinct (\emph{De Sensu} 6 446 b 28). Of course the dynamic in-formation is materially realized by the motions of bodies, but as Prichard observes, the only bodies in motion are ``the oscillation of certain particles'', the force with which the patterned disturbance propagates being communicated from one part of the medium to the next. The propagation of the patterned disturbance in the event of rarefaction shall be understood as a dynamic in-formation, not a kind of locomotion so much as a kind of alteration.

The second model is at work in Bacon's doctrine of the multiplication of the species (\emph{De multiplicatione specierum}). A body will cause a species, an image or likeness of it, to inhere, in some sense, in the medium adjacent to it. Moreover, species successively inhere in parts of the medium, each time causing the species to inhere in an adjacent part. This has the consequence that species are continuously generated along rectilinear paths that proceed in all directions, if unobstructed, from every point on the surface of a body. The propagation of the species does not involve locomotion, but the generation of the species multiplied in the medium. Species are not bodies and so are not subject to locomotion, understood as a change to a body’s location over time. So species, at least as Bacon conceives of them, do not fly through the air as Descartes (\emph{La dioptrique} AT VI 85) and Hobbes (\emph{Leviathan} 1 1 1) complained. The propagation of the species is, rather, the successive inherence of a form in different parts of the medium along rectilinear paths. And that is precisely how I am proposing to understand sonic propagation, as a kind of dynamic in-formation, specifically, as the successive inherence of a wave-form in different parts of the medium along rectilinear paths in every direction away from the source.

% section two_models_of_sonic_propagation (end)

\section{Motive and Objections} % (fold)
\label{sec:motive_and_objections}


Why identify sounds with events of rarefaction? What recommends The Wave Theory of Sound? The Wave Theory, or historical variants of it, have traditionally been motivated on phenomenological grounds. Broad nicely captures the emanative phenomenology of auditory experience that motivates The Wave Theory of Sound by contrasting auditory experience of sound with visual experience of color:
\begin{quote}
	But the noise is not literally heard as the occurrence of a certain sound-quality within a limited region remote from the percipient's body. It certainly is not heard as having any shape or size. It seems to be heard as coming to one from a certain direction, and it seems to be thought of as pervading with various degrees of intensity the whole of an indefinitely large region surrounding the centre from which it emanates. \citep[5]{Broad:1952kx}
\end{quote}
Colors are seen to inhere in surfaces of bodies located at a distance from the perceiver. Colors only ever inhere in spatially extended things and inherit the shapes and sizes of the extended things in which they inhere. But that is not how audition presents the sounds that we hear. Sounds are not experienced as qualties confined to a region remote from the perceiver. Nor are sounds experienced as qualities whose instances have the shape and size inherited from the extended things in which they inhere. Rather. sounds are heard to come from their sources. This is the first aspect of the emanative phenomenology of auditory experience, the directionality of audition. The directionality of audition is dynamic. It is the direction in which the patterned disturbance propagates along a rectilinear path from its source. In addition to the directionality of audition, Broad emphasizes the heard pervasiveness of sound: ``it seems to be \ldots\ pervading with various degrees of intensity the whole of an indefinitely large region surrounding the centre from which it emanates''. The heard pervasiveness of sound, suggests that not only is sound directional, but that it is multidirectional as well---that the patterned disturbance propagates in every direction, along rectilinear paths, away from the source. Moreover, the force with which the patterned disturbance propagates weakens as it diffuses through the medium as is evidenced by its being experienced as ``pervading with various degrees of intensity'' the whole of an indefinitely large region. The heard pervasiveness of sound is the hearer's auditory sense that they are presented with, at least from their partial perspective, an event of rarefaction. If the directionality of audition and the heard pervasiveness of sound have proved contentious, this is only because they have been misinterpreted.

The emanative phenomenology of audition, then, comprises, at least, the directionality of audition and the heard pervasiveness of sound. The emanative phenomenology of audition, so understood, naturally motivates The Wave Theory of Sound. For suppose that auditory experience does, in fact, have an emanative phenomenology, and that this aspect of our auditory experience is veridical, then in hearing sounds emanating from their sources would just be to hear an event of rarefaction. The direction from which the sound is coming is just the direction of the force with which the patterned disturbance propagates as it diffuses through the dense and elastic medium. \citet{Broad:1952kx} himself does not accept The Wave Theory of Sound because he does not accept that the emanative phenomenology of auditory experience is veridical. The stark phenomenological differences between vision, audition, and touch that he observes are, by Broad's lights, all undermined by the common causal mechanisms that underly the operation of the senses, and these phenomenological differences are at least misleading if not indeed illusory. At the very least, Broad's reflections establish that the emanative phenomenology of audition only motivates The Wave Theory of Sound if it is in fact veridical.

I shall consider two kinds of objections to The Wave Theory. One seeks to undermine the phenomenological motivation of The Wave Theory, the other seeks to draw our attention to aspects of the phenomenology of auditory experience that are allegedly inconsistent with The Wave Theory.

Here is O’Callaghan pressing the first kind of objection:
\begin{quote}
	It might be that sounds are heard to come from a particular place by being heard first to be at that place, and then to be at successively closer intermediate locations. But this is not the case with ordinary hearing. Sounds are not heard to travel through the air as scientists have taught us that waves do. \citep[34]{OCallaghan:2007xy}
\end{quote}
Notice that O’Callaghan understands the motion in a medium allegedly disclosed in auditory experience as a species of locomotion, as change of place over time. But the propagation of a patterned disturbance is better understood as a dynamic in-formation. Not the wave-form changing its location over time so much as differently located things, different parts of the medium, successively instantiating the wave-form. O'Callaghan is right that our auditory experience does not present sounds changing their place over time, like a sonic bullet. But that is to misdescribe the emanative phenomenology of audition. In hearing the sound coming from its source, the perceiver hears the direction with which the pattern disturbance propagates, understood as a dynamic in-formation, not a kind of locomotion so much as a kind of alteration.

If O’Callaghan’s objection focuses on the first aspect of the emanative phenomenology of audition, the directionality of audition---that sounds are heard to come from their sources, Pasnau’s objection focuses on the second aspect of the emanative phenomenology of audition, the heard pervasiveness of sound---that sounds are heard to pervade a volume. According to Pasnau, sounds are not heard to audibly fill a medium except in exceptional cases. Pasnau's objection turns on a misinterpretation of the heard pervasiveness of sound.

Pasnau claims that most sounds do not audibly fill the medium. So filling the medium must be something audibly accessible. Consider a brief sound, such as a call of one of London's feral parakeets. According to The Wave Theory, the sound is the propagation, in every direction, along rectilinear paths, of a patterned disturbance, through a dense and elastic medium. In one clear sense, the only audible aspect of this complex event is the patterned disturbance as it is through some interval of time. The outer boundary of the sphere, the narrow band which is the patterned disturbance, is audible in the sense of being a potential proximal cause of the auditory experience of the sound. So while the complex event may be envisioned as a growing sphere, since the sound is brief, the only audible aspect of the sound is at the moving boundary of the sphere, the narrow band which is the patterned disturbance. After all, if a perceiver is placed within the sphere between the source of the sound and the narrow band at the sphere's outer boundary, they are no longer in a position to hear the sound. 

In one clear sense that may be so, but there are other, relevant senses of audibility. So, if circumstances are propitious, we can hear the direction of the sound's propagation. We may even have a sense of how far off the source is. So aspects of the complex sound event are in another relevant sense audible and in that they are not merely confined to the patterned disturbance at the outer boundary of the sphere. Nor are these exceptional cases. 

The Wave Theory is only committed to sounds being heard to fill the air in this latter sense. In this sense, something is audible if it is heard in hearing a sound. Of course, even on the first sense of audible, understood as a potential proximal cause of the perception of the sound, a continuous sound, such as a sound of a waterfall, will audibly fill the air---the continuously produced patterned disturbances will pervade the space between the perceiver and its source. But as Pasnau observes, and The Wave Theory predicts, these are exceptional cases.

Pasnau not only presses the first kind of objection but also the second kind. That is, he objects not only to The Wave Theory’s phenomenological motivation, but maintains as well that there are aspects of the phenomenology of auditory experience that are inconsistent with it:
\begin{quote}
	We do not hear sounds as being in the air; we hear them as being at the place where they are generated. Listening to the birds outside your window, the students outside your door, the cars going down your street, in the vast majority of cases you will perceive those sounds as being located at the place where they originate. At least, you will hear those sounds as being located somewhere in the distance, in a certain general direction. But if sounds are in the air, as the standard view holds, then the cries of birds and of students are all around you. This is not how it seems (except perhaps in special cases ...). 
\end{quote}

Pasnau's argument has two parts:
\begin{enumerate}[(1)]
	\item A claim about where sounds are when we hear them if The Wave Theory is true.
	\item A claim that our auditory experience includes a distal element.
\end{enumerate}
For now bracket the fist claim, and consider only the second (we shall subsequently return to the issue of where the sounds would be if The Wave Theory were true). When one listens to the birds outside one’s window, the students outside one’s door, and the cars going down one’s street, what is it that one is listening to? A flat-footed answer would be, well, birds, students, and cars, or at least their audible activities. But birds, students, and cars, or at least their activities, while audible, are not themselves sounds but their sources, at least potentially. If we hear not only sounds, but their sources, then discovering a distal element in auditory experience is not sufficient to establish that the sounds that we hear are themselves distal. After all the distal element may be the presentation in auditory experience of not the sound but its source.

Another objection of the second kind that conflates sounds with their sources is O’Callgahan’s \citeyearpar[89]{OCallaghan:2007xy} argument from timbre. O’Callaghan argues that auditory constancies concerning timbre favor thinking of the sounds that we hear as the causing of the propagation of a patterned disturbance, as opposed to the propagation of the patterned disturbance, as The Wave Theory contends. After all, timbre, or a natural subcomponent of it, carries information about the material structure of bodies relevant to their participation in an event that causes the patterned disturbance to propagate, in every direction, along rectilinear paths, in a dense and elastic medium.  If timbre were an audible quality of sound, then this would favor thinking of sound as the causing of the propagation of a patterned disturbance. Timbre is an audible quality, to be sure, but is it best thought of as an audible quality of sound? The conclusion that O’Callaghan draws from Handel’s \citeyearpar{Handel:1995aa} research---that timbre depends, at least in part, upon features of the source and the characteristic manner in which it disturbs the medium---suggests, instead, that timbre, or at least some natural subcomponent of it, is better understood as an audible quality of the sound’s source, the sound-generating event.

If it is controversial what sounds are, how plausible is it to rely, as I have been doing, on the the distinction between sounds and their sources? After all, if sounds are up for grabs, isn’t the distinction between sounds and their sources up for grabs as well? What ought not to be controversial is that there are sound-generating events. While it may be controversial whether sound-generating events are audible, some latter-day Berkeleans such as A.D. Smith \citeyearpar{Smith:2002sa} deny it, it should be uncontroversial that such events exist. Which events they are will depend, of course, upon what sounds are since these events are the causes of sound, but it should be uncontroversial that there are such sources. But even allowing that the identification of an event as the cause of sound will depend upon the controversial issue of what sounds are, there are still things that we can conclude about sources, for example, that they can be at a distance from the perceiver, and that they have audible qualities if heard. Moreover, these conclusions about sources, if warranted, together with the claim that sources are audible, might legitimately ground the lines of criticism presently pursued. 

Where is the event of rarefaction? Its location is an occasion-sensitive matter (in Travis' \citeyear{Travis:2008la} sense). On some occasions, in some practical circumstances, the location of the event of rarefaction is perceiver-dependent. Sounds are located where they hear them (\citealt{Nudds:2009sf}, \citealt{OShaughnessy:2009aa}). If sounds are events of rarefaction, then sounds, on such occasions, would count as located at the intersection between the outer band at the boundary of the expanding sphere and the perceiver. On other occasions, in other practical circumstances, a perceiver-independent location of sound is required. Perhaps no perceiver is relevant, or perhaps no one heard the sound. In such circumstances where should the sound be located? We cannot locate sounds in the region encompassed by their boundaries as they lack stable and determinate boundaries, and given the neat symmetry of the event, the patterned disturbance propagating in very direction from its source, it is natural to locate them at their center, at or near their source (see Sorenson's \citeyear{Sorensen:2009aa} analogy with earthquakes, another kind of event of rarefaction).

If the location of sound, conceived as an event of rarefaction, is an occasion-sensitive matter, then considered in abstraction from any practical reason for saying so, sounds lack determinate locations. If The Wave Theory of sound is a coherent credible alternative in the metaphysics of sound, we should reject any taxonomy of sound based on its determinate location. Thus \citet{Casati:2014hw} taxonomizes non-aspatial views of sound as distal, medial, or proximal, depending upon their determinate location. According to The Wave Theory, sounds are events with location and so are not aspatial. Nor are they distal, medial, or proximal, since their location is not determinately distal, medial, or proximal. The taxonomy should be rejected since it unreasonably excludes a live possibility.


% section motive_and_objections (end)

\section{The Event of Rarefaction} % (fold)
\label{sec:the_event_of_rarefaction}

Allow me, in this section, to make a summary statement of the position at which we have arrived.

According to The Wave Theory, as developed herein, the propagation of a patterned disturbance, in every direction, along rectilinear paths, through a dense and elastic medium is the progressive instantiation of a wave-form, a kind of dynamic in-formation, realized by the motion of the local parts of the medium, the oscillation of certain particles. Though the sound event may be said to have location, sonic propagation is not best modelled on the locomotion of a body, like a sonic missile. As O’Callaghan observes, that is not how auditory experience presents sound as coming from its source. It is rather better understood as a kind of dynamic in-formation. Sounds are heard to come from their sources in the sense that a rectilinear direction of the propagation of the patterned disturbance in the growth of the sound event is disclosed in auditory experience. On that model, there are certain natural alternative understandings of the location of a sound event. Locating the sound event in the space encompassed by stable and determinate boundaries is not possible since they lack these. On certain occasions, for certain practical purposes, sounds may be said to be where we hear them. On other occasions, for other purposes, sounds may be said to be located at their center, at or near their sources. And each alternative is consistent with the sound event being the propagation, in every direction, along rectilinear paths, of a patterned disturbance, through a dense and imperfectly elastic medium understood as the progressive instantiation of a wave-form realized by the motion of the local parts of the medium.

As a dynamic in-formation, the sound event has a kind of unity irreducible to the motion of the local parts of the in-formed medium. Conceiving of the propagation of sound on the model of the locomotion of a body---a sonic missile---mistakes the unity of the sound event for the unity of a body. Sound events may lack the unity of a body. After all, events and bodies have different modes of being. But sound events nevertheless possess sufficient unity to distinguish them from the in-formed medium that they existentially depend upon. It is a dynamic unity (on dynamic principles of unity see \citealt{Johnston:2006js}). It is the force with which the patterned disturbance propagates, in every direction, along rectilinear paths, that explains the growth of the sound event. While the sound event may be realized by the motion of the local parts of the medium, the oscillation of certain particles, it is the force of its propagation, communicated from one part to the next, that determines the dynamic in-formation. The sound event is realized by the motion of the local parts of the medium without reducing to such motion because of its dynamic unity, the force with which it grows in the dense and imperfectly elastic medium. And it is the direction of this force that is disclosed, more or less clearly, in the emanative phenomenology of auditory experience.

% section the_event_of_rarefaction (end)

\section{Hearing Sources in the Sounds They Generate} % (fold)
\label{sec:hearing_sources_in_the_sounds_they_generate}

We hear sounds. If circumstances are propitious, we can hear as well their sources (or so we have been assuming). At least two questions arise. Assuming that audition has a function, is the function of audition to afford the perceiver with auditory awareness of sounds or their sources? And what is the relationship, in audition, between the sounds that we hear and the sources that we hear when we do hear them?

Consider first the function of auditory awareness. Proximal perturbations in the dense and elastic medium, be it air, water, or some other suitable material, are relatively uninteresting features of the natural environment, at least from a biological point of view. The distal causes of these proximal perturbations, however, may be of vital concern, be they the approach of predator or prey. Given the selective advantage afforded by hearing the distal sources of the proximal perturbations impinging upon the perceiver, it is natural to think that the function of audition is to afford auditory awareness of the sources of sounds. This is an empirical bet. But it is an empirical bet shared by other philosophers and psychologists (\citealt{Nudds:2009sf}, \citealt{Bregman:1990aa})

What is the relationship, in audition, between the sounds that we hear and the sources that we hear when we do hear them? We shall consider two models. The first is more familiar, but if we assume the truth of The Wave Theory of Sound, the second is more likely to be true. 

On the first model, we hear sources by hearing the sounds that they generate. Sounds are audible. Moreover, sounds are audible in themselves. Sounds are audible in themselves in that they contain within themselves the power of their own audibility. And this is understood as sounds being the immediate objects of the explicit awareness afforded by auditory experience. And we mediately hear a source by immediately hearing the sound that it generates. We only come to be aware of the source by first being aware of the sound that it generates. And, as a consequence, sources, on this model, are audible, but not audible in themselves, but audible only in virtue of sounds that are audible in themselves. Sources, so understood, are the mediate objects of the explicit awareness afforded by auditory experience. Structurally at least, the first model, then, is a kind of sonic indirect realism about heard sources.

The first model is, in an important respect, incomplete. For how do sounds mediately present their sources when they do? What is the audible relation between the sound and its source that would explain how the presentation, in audition, of a sound, can, at the same time, be the mediate presentation of its source? Causal relations, the relation of inherence that a quality bears to its \emph{substratum}, and the part--whole relation have been discussed by \citet{Leddington:2014aa} and \citet{OCallaghan:2011az}. However, as we shall see, no matter how it is best completed, the first model misdescribes the structure of auditory experience.

On the second model, we do not hear sources by hearing the sounds that they generate. Rather we hear the sources of the sound, when we do, through or in the sounds that they generates (\citealt{Leddington:2014aa}, \citealt[chapter 4]{Kalderon:2018oe}). When we hear a source, the source is immanent in the sound it is heard to generate. On the second model, sources need not be mediate objects of audition. We may be explicitly aware of the source of a sound without first being explicitly aware of its sound. The second model, then, rejects the indirect sonic realism of the first.

The first model faces phenomenological and empirical problems.

Sonic indirect realism does not seem an apt description of our auditory experience. In ``The Origin of the Work of Art'' Heidegger presents an opposing view:
\begin{quote}
	We never really first perceive a throng of sensations, e.g., tones and noises, in the appearance of things \ldots\ ; rather we hear the storm whistling in the chimney, we hear the three-motored plane, we hear the Mercedes in immediate distinction from the Volkswagen. Much closer to us than all sensations are the things themselves. We hear the door shut in the house and never hear acoustical sensations or even mere sounds. \citep{Heidegger:1935uq}
\end{quote}
Nothing hangs on Heidegger’s apparent acceptance of the empiricist identification of sound with acoustical sensation. What is important is Heidegger’s denial of the claim that we hear the source of a sound by attending to that sound. Rather, we hear the source without first hearing its sound. In hearing the storm whistling in the chimney, the three-motored plane, the Mercedes in distinction from the Volkswagen, we are explicitly aware of the sources of sound without being explicitly aware of the sounds that they generate. That is consistent with maintaining that hearing a source necessarily involves acoustical sensation. We may be implicitly aware of the sound in explicitly attending to its source. And yet Heidegger is clearly denying the claim that he hear the source of a sound by hearing the sound. There is one explicit experience, hearing the storm whistling in the chimney, and no distinct explicit experience of its sound, even if hearing the source involves implicit awareness of its sound. If Heidegger is right about the phenomenology, we should reject the first model of the relationship between heard sounds and their sources. (For discussion see \citealt{Leddington:2014aa} and \citealt[chapter 3]{Kalderon:2018oe})

\citet{Nudds:2009sf}, drawing on the work of \citet[chapter 4]{Bregman:1990aa}, raises an empirical issue affecting the first model. In circumstances with multiple active sources of sounds, the auditory system faces a familiar underdetermination problem. The proximal stimuli consist in frequency components, and the auditory system must group these frequency components into the separable sounds that are heard. However, no particular grouping of frequency components is uniquely determined by the proximal stimuli. How does the auditory system address this undertermination problem? It does so, in part, by making a number of substantive assumptions about the likely source of these frequency components. That is to say, that the auditory system segments a sound from all that is heard by identifying its likely source. At least in the special circumstances where there are multiple active sources of sound, the sounds that we hear are not heard independently of hearing their source. But if that is right, then sounds are not audible in themselves, at least not in every circumstance. They do not contain within themselves the power of their own audibility, at least not wholly, in every circumstance. If anything, something like the reverse is true in the given circumstance. In circumstances with multiple active sources of sound, sounds are audible, but not audible in themsleves, but audible only insofar as one hears the sources that generate them. (If the sources remain elusive to the auditory system, one wouldn't hear the several sounds but only an indiscriminate noise.) 

Sounds need not be audible in themselves. Rather, in some cases at least, they are more like audible media. What does it mean to describe sounds as perceptual media? Perceptual media need not be thought of as physical media, the movement of whose local parts, ``the oscillation of certain particles,'' realize the progressive instantiation of a wave-form. While the idea of physical media merely answers to the demands of being a causal intermediary, the idea of perceptual media answers to the demands of perceptual accessibility. Light does not require physical media in which to propagate in the way that sound waves do. As the Michelson–Morley experiment of 1887 went some way toward showing, there is no Luminiferous aether. But the illuminated air may be a perceptual medium, nonetheless. So consider the following. Just as illumination makes the visible perceptually accessible, sound makes the activities of distal bodies perceptually accessible. Without illumination, the colors of distal bodies remain unseen, without sound, the activities of distal bodies remain unheard. One sees through, or in, illuminated media, such as air or water, and thereby perceives the colors of distal bodies arrayed in the natural environment. One hears through, or in, audible media, the sound, and thereby perceives the activities of distal bodies arrayed in the natural environment.

% By means of the propagation of light waves, the visible aspects of distal bodies are seen. By means of the propagation, in every direction, along rectilinear paths, of the patterned disturbance through a dense and imperfectly elastic medium, that is, by means of sound, the audible activities of distal bodies are heard.

Perceptual media like sound and the illuminant, are perceptible. Moreover, not only are perceptual media themselves perceptible, but they are perceptible in a certain way. Specifically, they are not perceptible in themselves, but owe their perceptibility to other things which are perceptible in themselves, the objects the perceptual media make perceptually accessible. One sees the character of the illumination by seeing the way objects are illuminated. When viewing a brightly lit pantry, one sees the brightness of the pantry by seeing the brightly lit objects arranged in it. So the illuminant is visible, though not visible in itself, but owes its visibility to the objects that it illuminates. (For a comparison with Aristotle’s definition of transparency, \emph{De anima} 2 7 418 b 4–6, see \citealt[41--42]{Kalderon:2015fr}; for a contemporary defence of this claim see \citealt{Hilbert:2007qy}.) Like the illuminant, sound is perceptible, though perceptible in a certain way. In cases where one hears a sound and its source, one hears the character of a sound by hearing the activities of its distal source. (Think of how difficult it is to describe ecological sound without describing audible aspects of its source.) In such cases, sound is audible, though not audible in itself, but owes its audibility to the distal source that it discloses.

The disclosure of the distal source of sound in auditory experience may require the activity of an attentive listener. I turn, and listen, and hear the call of the feral parakeet. What I hear is the audible activity of a distal body, the animate body of the feral bird. Audition affords me explicit awareness of the parakeet’s call. I hear how loud it is, its distinctive timbre, and its sharpness and urgency. I hear the parakeet's call through, or in, the sound that it makes. The parakeet's calling generates a patterned disturbance that propagates, in every direction, along rectilinear paths, through the dense and imperfectly elastic air. It is through, or in, this audible media, the sound that it makes, that the call of the feral parakeet is heard. In turning, and listening, and hearing, I alter my auditory perspective on the natural environment to bring an aspect of that environment, the audible activity of the feral parakeet, into earshot. Turning, and listening, and hearing---actively changing my auditory perspective on the natural environment---is a sympathetic response to what is heard. Changing my auditory perspective to increase the acuity with which the feral parakeet is heard is to sympathetically respond to the call of the feral parakeet. Preparedness to act in certain ways so that the impingement of the force of the propagation of the patterned disturbance, the dynamic principle of unity of the sound event, carries information about its distal source, sensitivity to which constitutes, in propitious circumstances, explicit auditory awareness of that source, is what makes possible the presentation, in auditory experience, of the source of the sound. 

% section hearing_sources_in_the_sounds_they_generate (end)

% nocite
\nocite{Shields:2016ix}
\nocite{Hett:1936fk}
\nocite{Locke:1975pi}
\nocite{Lindberg:1998dy}
\nocite{Hobbes:1904hw}
\nocite{Descartes:1985df}
\nocite{Berkeley:1734fk}

% Bibligography
\bibliographystyle{plainnat} 
\bibliography{Philosophy} 

\end{document}