%!TEX TS-program = xelatex 
%!TEX TS-options = -synctex=1 -output-driver="xdvipdfmx -q -E"
%!TEX encoding = UTF-8 Unicode
%
%  rarefaction
%
%  Created by Mark Eli Kalderon on 2018-03-29.
%  Copyright (c) 2018. All rights reserved.
%

\documentclass[12pt]{article} 

% Definitions
\newcommand\mykeywords{audition, sound, source, wave theory}
\newcommand\myauthor{Mark Eli Kalderon}
\newcommand\mytitle{The Event of Rarefaction}
\newcommand\mysubtitle{A Defence and Development of The Wave Theory of Sound}

% Packages
\usepackage{geometry} \geometry{a4paper} 
% \usepackage{txfonts}
% \usepackage{enumerate}
% \usepackage{setspace}
% \doublespace % Uncomment for doublespacing if necessary
% \usepackage{epigraph} % optional

% XeTeX
\usepackage[cm-default]{fontspec}
\usepackage{xltxtra,xunicode}
\defaultfontfeatures{Scale=MatchLowercase,Mapping=tex-text}
\setmainfont{Hoefler Text}

% Bibliography
\usepackage[round]{natbib}

% Title Information
\title{\mytitle}
\author{\myauthor}
\date{} % Leave blank for no date, comment out for most recent date

% PDF Stuff
\usepackage[plainpages=false, pdfpagelabels, bookmarksnumbered, backref, pdftitle={\mytitle}, pdfauthor={\myauthor}, pdfkeywords={\mykeywords}, xetex, unicode=true]{hyperref} 

%%% BEGIN DOCUMENT
\begin{document}

% Title Page
\maketitle
% \vskip 2em \hrule height 0.4pt \vskip 2em
% \epigraph{} % optional; make sure to uncomment \usepackage{epigraph}

% Layout Settings
\setlength{\parindent}{1em}

% Main Content

\section{Motion in a Medium} % (fold)
\label{sec:motion_in_a_medium}

In this paper I shall defend and develop The Wave Theory of Sound. The Wave Theory has come on hard times of late---unfairly, to my mind. Many are of the opinion that it is no longer a viable alternative in the metaphysics of sound, and this despite a long and venerable history. As we shall see, many apparent difficulties facing The Wave Theory are due to an exclusive focus on hearing sounds. An exclusive focus on hearing sounds can lead to a distorted conception not only of audition but of the sounds themselves. With respect to audition it can encourage the thought that the function of audition is to afford the perceiver with auditory awareness of sounds. But if we can hear not only sounds, but their sources, understood as sound-generating events, it is more likely that the function of audition is to afford the perceiver awareness of the distal sources of sound. With respect to sound, an exclusive focus on hearing sounds can lead to conflating features of sources of sound with features of the sounds themselves. Many of the recent criticisms of The Wave Theory turn on just such a conflation.

% section motion_in_a_medium (end)

\section{Assumptions} % (fold)
\label{sec:section_name}

In defending and developing The Wave Theory of Sound, I shall make a number of assumptions:
\begin{enumerate}
	\item Sounds are particulars.
	\item Sounds are the bearers or \emph{substrata} of audible qualities.
	\item Sounds are not the sole bearers or \emph{substrata} of audible qualities; in propitious circumstances, we can hear, as wel, their sources (understood as sound-generating events).
	\item (With Berkeley) if sources are audible, then sources have audible qualities.
	\item The bearers of audible qualities are events, and hence audible qualities are essentially sustained by activity.
\end{enumerate}
Though I cannot here make a full dress defence, allow me to both explain and motivate these insofar as I can.

% section assumptions (end)

\section{Sonic Events} % (fold)
\label{sec:sonic_events}

If sounds are events, then what events are they? Consider the following three candidates:

\begin{enumerate}
	\item Sounds are events that would cause a patterned disturbance to propagate, in every direction, through a dense and elastic medium should there be one. \citep{Casati:1994aa}
	\item Sounds are the causing of a patterned disturbance to propagate, in every direction, through a dense and elastic medium. \citep{OCallaghan:2007xy}
	\item Sounds are events of rarefaction. Specifically, they are the propagation, in every direction, of a patterned disturbance through a dense and elastic medium. (The Wave Theory, \citealt{OShaughnessy:2009aa}, \citealt{Sorensen:2009aa}, \citealt[chapters 3–4]{Kalderon:2018oe})
\end{enumerate}

% section sonic_events (end)

\section{Two Models of Sonic Propagation} % (fold)
\label{sec:two_models_of_sonic_propagation}

Plato and Aristotle identify sound with motion in a medium (\emph{Timaeus} 67a–c, 80a–b; \emph{De Anima} 2 8). Though they identify sound with motion in a medium we are still a long way from The Wave Theory as defended herein (see, for example, Cornford's \citeyear[320 n.1]{Cornford:1935fk} criticism of \citealt{Archer-Hind:1888qd}). For one thing, both Plato and Aristotle conceive of the medium being moved as a continuous unity (on this point, in the \emph{Timaeus} account of sound, see \citealt[109]{Beare:1906uq}). Moreover, the view may be differently developed depending upon what, exactly, motion---\emph{kìnêsis}---means. \emph{kìnêsis} is Aristotle's general term for change of any kind, and so not merely locomotion. And though it is not a technical term in Plato's writing the way that it is Aristotle's, a similar usage can be found especially in the \emph{Timaeus}. This is important since \emph{kìnêsis} here could be locomotion, a change of place over time, or it could instead be a species of alteration. These contrasting interpretative hypotheses provide us with two models of sonic propagation.

On the first model, sonic propagation is understood in terms of locomotion. So understood, the patterned disturbance is literally travelling through the dense and elastic medium. Locomotion is a change paradigmatically undergone by bodies. In the case of Plato and Aristotle, the relevant bodies would be continuous bodies of air produced by a sharp percussive blow. However, since we are assuming that sounds are events and not bodies, does that mean that the model is inapplicable from the start? The issue is unobvious since events have locations, though perhaps not in the way that bodies do (see, for example, \citealt{Davidson:1969da}) And, sometimes at least, events change their locations over time. Thus the fight started in the bar and spilled into the street, and the conga line started in the dining room and wound its way into the living room. Still, if it is specifically the patterned disturbance that is in ``motion'', and not the dense and elastic medium that instantiates the patterned disturbance, then the propagation of the patterned disturbance is better understood as a species of alteration.

For this reason, Prichard denies that waves and sounds, being what they are, are subject to locomotion. Only bodies move, and waves and sounds are not bodies:
\begin{quotation}
	But \dots\ I also made the same remark (viz.\ that only a body could move) to a mathematician here. What was in my mind was that it is mere inaccuracy to say that a wave could move, and that where people talk of a wave as moving, say with the velocity of a foot, or a mile, or 150,000 miles, a second, the real movement consisted of the oscillations of certain particles, each of which took place a little later than a neighbouring oscillation.
	
	He scoffed for quite a different reason. He said that you could illustrate a movement by a noise---that, for example, if an explosion occurred in the middle of Oxford the noise would spread outwards, being heard at different times by people at varying distances from the centre, so that at one moment the noise was at one place and that a little later it was somewhere else, and in the interval it had moved from one place to the other.
	
	Now, of course, it was not in dispute that in the process imagined people in different places each heard a noise at a rather different time. The only question was, `Was the succession of noises a movement?', and I think that on considering the matter you will have to allow that it was not, and that what happened was that he, being certain of the noises, and wanting to limit the term `movement' to something he was certain of, used the term `movement' to designate the succession of noises, implying that this was the real thing of which we were both talking. But if this is what happened, then he was using the term `movement' in a sense of his own, and in saying that in the imagined case he was certain of a movement, he was being certain of something other than the opposite of what I was certain of. \citep[99]{Prichard:1950kx}
\end{quotation}
\citet[430 n. 29, appendix,]{Burnyeat:1995fk} lampoons Aristotle for making similar claims by citing Prichard echoing them here. I have argued, that at least in this instance, Burnyeat is hoisted by his own petard, \citep[Chapter 3.2]{Kalderon:2015fr}. If Prichard is right, then sonic propagation cannot be understood in terms of locomotion.

On the second model, sonic propagation is understood, not in terms of locomotion, but in terms of alteration. So understood, sonic propagation is a kind of dynamic in-formation. Specifically, the patterned disturbance is successively in-forming differently located parts of the dense and elastic medium along a rectilinear path in a direction away from the source of the sound. Thus the source's sounding will initially produce a patterned disturbance in the dense and elastic medium adjacent to the source. At a later moment, the patterned disturbance will be instanced by a different part of the dense and elastic medium, further away from the source along a rectilinear path. On this model, the patterned disturbance is not travelling, at least not literally. It is not the case that the patterned disturbance is now located here and now there. Rather, differently located parts of the dense and elastic medium come to successively instantiate the patterned disturbance. A change of state and travel are distinct (\emph{De Sensu} 6 446 b 28)

% section two_models_of_sonic_propagation (end)

\section{Motive and Objections} % (fold)
\label{sec:motive_and_objections}

\begin{quote}
	But the noise is not literally heard as the occurrence of a certain sound-quality within a limited region remote from the percipient's body. It certainly is not heard as having any shape or size. It seems to be heard as coming to one from a certain direction, and it seems to be thought of as pervading with various degrees of intensity the whole of an indefinitely large region surrounding the centre from which it emanates. \citep[5]{Broad:1952kx}
\end{quote}


% section motive_and_objections (end)

\section{The Event of Rarefaction} % (fold)
\label{sec:the_event_of_rarefaction}



% section the_event_of_rarefaction (end)

% nocite
\nocite{Shields:2016ix}
\nocite{Hett:1936fk}

% Bibligography
\bibliographystyle{plainnat} 
\bibliography{Philosophy} 

\end{document}